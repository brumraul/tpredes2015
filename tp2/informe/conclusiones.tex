\section{Conclusiones}

Tras haber corrido los distintos test sobre los host bajo estudio notamos que no obtuvimos variaciones en las rutas utilizadas por los paquetes, sino que generalmente se mantuvieron las mismas rutas entre las distintas corridas. 
Notamos también que la información provista por los servicios de geolocalización en muchos casos no se se corresponde con la localización física del router.\\
En cuanto a la localización de enlaces submarinos mediante el test de hipótesis realizado en algunos casos detecto correctamente los mismos pero otros como ser el caso de la universidad de Pretoria si bien detecto los enlaces submarinos de forma correcta también arrojo otros outliers que no se corresponden con enlaces submarinos. En vista de estos resultados concluimos que el uso de esta herramienta no sería viable para la detección de los enlaces submarinos, si podría utilizarse para acotar los posibles candidatos pero deberíamos utilizar alguna otra herramienta como apoyo para evitar estos falso positivos.\\
Otra particularidad que notamos es que muchos routers asignan una menor prioridad a los paquetes ``ICMP'' lo que origino que varios saltos tuviesen $\Delta$RTT negativos.

A continuación expondremos las particularidades de cada caso de estudio:

\subsection{Universidad de Tokyo}

Para este caso de estudio mas allá del error informado por los servicios de geolocalización para el salto 10 de la tabla de la figura \ref{table:tokyo}, tanto los valores de RTT obtenidos como los outliers detectados por el test de Grubbs parecen tener correspondencia con lo realmente acontecido.\\
Si bien los $\Delta$RTT obtenidos parecen no seguir una distribución normal los outliers obtenidos mediante el test de Grubbs para detectar los enlaces submarinos parecen ser los correctos y coinciden con lo que se puede apreciar en los gráficos de las figura \ref{histo:tokyo} y la figura \ref{lines:tokyo}.

\subsection{Universidad de Pretoria}
Este caso de estudio se eligió originalmente para obtener la ruta a un continente diferente a los otros casos elegidos, pero al realizar las tareas de geolocalización nos encontramos con que el servidor de esta universidad sudafricana se encuentra en Inglaterra. Aún así se decidió no cambiar el caso dado que el correspondiente a España tiene un origen diferente, lo que nos permite ver la diferencia entre ambas rutas que tienen un destino cercano. \\
En este caso de estudio se puede ver que con el test de Grubbs se detectaron cuatro outliers, pero siendo dos de ellos falsos positivos como se pueden ver que son el salto 2 y el 9, lo cual se puede ver claramente en los gráficos de las figura \ref{histo:pretoria} y la figura \ref{lines:pretoria}. Aún así devolvió correctamente los correspondientes a enlaces submarinos entre Buenos Aires y Miami, y entre Miami y París.

\subsection{Universidad de Málaga}
Este caso a diferencia de los otros el trace tiene un origen diferente (USA) quizás por eso la ruta parece más directa.
Si bien la distribución según el test no es normal, al localizar los outliers con el test de Grubbs se pudo encontrar el único enlace continental en dicho path. Esto quedo claramente evidenciado en la figura \ref{histo:malaga} y la figura \ref{lines:malaga}.
Otro detalle menor es que los RTT son bastante más chicos comparados con los de los otros traces, esto quizás se deba a que es otro ISP en otro país.



\subsection{Universidad MIT}
Si bien al correr los test para esta universidad nos dimos cuentas que el servidor que contenía el sitio web se encontraba en Santiago de Chile en lugar de Estados Unidos por ser provisto por Akamai, decidimos dejarlo ya que por el mal ruteo de claro igualmente atravesamos enlaces transcontinentales.\\
La herramienta implementada detectó correctamente outliers en los tiempos de las rutas. y como se comentó anteriormente, estos correspondieron a enlaces submarinos.\\ 
Un dato llamativo en este caso de estudio es la gran cantidad de ip privadas que aparecen en las rutas. Estas direcciones ip privadas son conexiones entre routers configuradas como redes internas, configuradas de esa manera por el ISP.\\
Algo para notar también es el horrible routeo que posee el isp claro ya que para ir a un host de un país limítrofe como Chile primero pasa por Estados Unidos y España para llegar hasta el host de destino incrementando el RTT de la conexión notablemente.

