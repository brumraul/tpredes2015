\section{Conclusiones}

Tras haber corrido los distintos test sobre los host bajo estudio notamos que no obtuvimos variaciones en las rutas utilizadas por los paquetes, sino que generalmente se mantuvieron las mismas rutas entre las distintas corridas. 
Notamos también que la información provista por los servicios de geolocalización en muchos casos no se se corresponde con la localización física del router.\\
En cuanto a la localización de enlaces submarinos mediante el test de hipótesis realizado en algunos casos detecto correctamente los mismos pero otros como ser el caso de la universidad de Pretoria si bien detecto los enlaces submarinos de forma correcta también arrojo otros outliers que no se corresponden con enlaces submarinos. En vista de estos resultados concluimos que el uso de esta herramienta no sería viable para la detección de los enlaces submarinos, si podría utilizarse para acotar los posibles candidatos pero deberíamos utilizar alguna otra herramienta como apoyo para evitar estos falso positivos.\\
Otra particularidad que notamos es que muchos routers asignan una menor prioridad a los paquetes ``ICMP'' lo que origino que varios saltos tuviesen $\Delta$RTT negativos.

A continuación expondremos las particularidades de cada caso de estudio:

\subsection{Universidad de Tokyo}

Para este caso de estudio mas allá del error informado por los servicios de geolocalización para el salto 10 de la tabla de la figura \ref{table:tokyo}, tanto los valores de RTT obtenidos como los outliers detectados por el test de Grubbs parecen tener correspondencia con lo realmente acontecido.\\
Si bien los $\Delta$RTT obtenidos parecen no seguir una distribución normal los outliers obtenidos mediante el test de Grubbs para detectar los enlaces submarinos parecen ser los correctos y coinciden con lo que se puede apreciar en los gráficos de las figura \ref{histo:tokyo} y la figura \ref{lines:tokyo}.

\subsection{Universidad de Pretoria}

\subsection{Universidad de Málaga}
Este caso a diferencia de los otros el trace tiene un origen diferente (USA) quizás por eso la ruta parece más directa.
Si bien la distrubución según el test no es normal, al localizar los outliers con el test de Grubbs se pudo encontrar el único enlace continetal en dicho path. Esto quedo claramente evidenciado en la figura \ref{histo:malaga} y la figura \ref{lines:malaga}.
Otro detalle menor es que los RTT son bastante más chicos comparados con los de los otros traces, esto quizás se deba a que es otro ISP en otro país.



\subsection{Universidad MIT}
La herramienta implementada detectó correctamente outliers en los tiempos de las rutas. y como se commentó anteriorente, estos correspondieron a enlaces submarinos. En cuanto a la ruta tomada por los paquetes enviados hacia el servidor con la página del MIT, podemos notar que la misma no es muy directa, saltando de servidores en EEUU hacia España, luego pasando por Chile para llegar nuevamente hacia EEUU. 
